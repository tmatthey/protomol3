%----------------------------------
%
%  MTREM notes
%
%----------------------------------

% PACKAGES -----------------------------------
\documentclass[a4paper,11pt,oneside]{article}
\usepackage{epsf,amssymb}
\setlength{\parskip}{6 pt}
\usepackage{epsfig}
% --------------------------------------------------------------

\newtheorem{theorem}{\sc Theorem}[section]

\newtheorem{defi}[theorem]{\sc Definition}

\newenvironment{definition}{\begin{defi} \rm}
{\hfill $\Box$
\end{defi}}


% PAGE SETUP/LINE SPACING/MARGINS ------------------------------%
\setlength{\textwidth}{6.0in} \setlength{\textheight}{8.5in}
\setlength{\topmargin}{0in} \setlength{\oddsidemargin}{0.0in}
\setlength{\evensidemargin}{0.0in}
\vfuzz1pc               % Don't report overfull v-boxes if over-edge is < 1pc
\hfuzz1pc               % Don't report overfull h-boxes if over-edge is < 1pc
%\renewcommand{\baselinestretch}{1.5}    % Genuine double spacing if 2.0
%%\allowdisplaybreaks[3]  % Split displayed equations at the end of a page with a tolerance of 1,2,3 or 4
% My additions
\renewcommand{\topfraction}{0.85}
\renewcommand{\textfraction}{0.1}
\renewcommand{\floatpagefraction}{0.75}
% --------------------------------------------------------------

%\bibliographystyle{siam}
\bibliographystyle{plain}   % Uses plain bibtex style.

% MATHS NOTATION -----------------------------------------------------
\def\la{\langle}                %  left bracket for presentation
\def\ra{\rangle}                %  right bracket for presentation
\def\lb{\lbrace}                %  left curly bracket
\def\rb{\rbrace}                %  right curly bracket
\def\id{\equiv}                 %  identical words
\def\pad{\$}                    %  padding symbol
% Greek Letters ------------------------------------------------------
\def\a{\alpha}         % alpha
\def\b{\beta}          % beta
\def\d{\delta}         % delta
\def\e{\epsilon}       % empty word
\def\q{\theta}         % theta
\def\p{\phi}           % phi
\def\D{\Delta}         % big delta
\def\P{\Psi}           % big psi
\def\Half{\frac{1}{2}}
% HYPHENATION --------------------------------------------------------
\hyphenation{semi-group} \hyphenation{sub-semi-group}
\hyphenation{fin-ite-ly}
% THEOREMS AND NUMBERING ---------------------------------------
\newtheorem{lemma}[theorem]{Lemma}
\newtheorem{proposition}[theorem]{Proposition}
\newenvironment{proof}[1][Proof]{\begin{trivlist}
\item[\hskip \labelsep {\bfseries #1}]}{\end{trivlist}}
% --------------------------------------------------------------------

\title{Normal Mode notes.}

\def\qed{\hfill\(\Box\)\smallskip}

\begin{document}

%\maketitle

\pagebreak
%****Main CV************************************************************************************************
\section*{Notes on SCPISM.}
In the following discussion paramiters
$R_{i,COV},~\delta(+/-),~\gamma,~R_p,~R_{i,vdW},~A,~B,~C$ and $E$
are provided for each atom type. The self energy is dependent on the
inter-atom distance which is denoted, for atoms $i$ and $j$,
$r_{ij}=||\mathbf{r}_j-\mathbf{r}_i||$ for atomic position vectors
$\mathbf{r}_i$ and $\mathbf{r}_j$.

\subsection*{Born radius and its first derivatives.}
We can write the Born radius defined by Hassan, for non-polar atoms,
as
\begin{equation}
R_i = \zeta_i + \eta_i\sum_{j\ne i}^{N}f(r_{ij})\exp(-C_ir_{ij}),
\end{equation}
for
\begin{equation}
\zeta_i = R_{i,COV}+\delta(+/-)+\gamma - \frac{\gamma
A_i}{4\pi(R_p+R_{i,vdW})^2},~\eta_i~=~\frac{\gamma
B_i}{4\pi(R_p+R_{i,vdW})^2},
\end{equation}
and
\begin{equation}
f(r_{ij}) = \left\{
\begin{array}{c c}
\left(1-\frac{r_{ij}^4}{625}\right)^4 & 0\le r_{ij}<5,\\
0 & r_{ij}\ge 5.
\end{array}
\right.
\end{equation}

For polar Hydrogens we have an additional term to give the modified Born radius
\begin{equation}
R_i^{H+} = R_i + \sum_{j\ne i}^M g_ig_jf(r_{ij})\exp(-E_ir_{ij}).
\end{equation}

For non-polar atoms the derivative of the Born radius w.r.t. the
atom pair distance is
\begin{equation}
\frac{\mathrm{d}R_i}{\mathrm{d}r_{ij}} =
\eta_i\left(\frac{\mathrm{d}f}{\mathrm{d}r_{ij}}(r_{ij})-f(r_{ij})C_i\right)\exp(-C_ir_{ij}),
\end{equation}
with (for $r_{ij} < 5$),
\begin{equation}
\frac{\mathrm{d}f}{\mathrm{d}r_{ij}}(r_{ij}) =
-\frac{16}{625}r_{ij}^3\left(1-\frac{r_{ij}^4}{625}\right)^3.
\end{equation}

For polar Hydrogens
\begin{equation}
\frac{\mathrm{d}R_i^{H+}}{\mathrm{d}r_{ij}} =
\frac{\mathrm{d}R_i}{\mathrm{d}r_{ij}} +
g_ig_j\left(\frac{\mathrm{d}f}{\mathrm{d}r_{ij}}(r_{ij})-f(r_{ij})E_i\right)\exp(-E_ir_{ij}).
\end{equation}

\subsection*{Self energy and its first derivatives.}
The self energy term, from Hassan, is
\begin{equation}
E_s =
\frac{1}{2}\sum_{i=1}^N\frac{q_i^2}{R_i}\left(\frac{1}{D_s(R_i)}-1\right),
\end{equation}
where the screening function is given by
\begin{equation}
D_s = \frac{1+\epsilon_s}{1+k\exp(-\alpha_iR_i)}-1,
\end{equation}
where $\epsilon_s$ is the dielectric constant of the bulk solvent (assumed to
be 80). Note in Hassans paper $k=(\epsilon_s-1)/2$.

The first derivative of the self energy is
\begin{eqnarray}
\nonumber\frac{\delta E_s}{\delta r_{ij}} &=&
\frac{1}{2}\frac{q_i^2}{R_i^2}\frac{\mathrm{d}R_i}{\mathrm{d}r_{ij}}\left(1-\frac{1}{D_s(R_i)}-\frac{R_i}{D_s^2(R_i)}\frac{\mathrm{d}D_s}{\mathrm{d}R}(R_i)\right)+\\
&&\quad\frac{1}{2}\frac{q_j^2}{R_j^2}\frac{\mathrm{d}R_j}{\mathrm{d}r_{ij}}\left(1-\frac{1}{D_s(R_j)}-\frac{R_j}{D_s^2(R_j)}\frac{\mathrm{d}D_s}{\mathrm{d}R}(R_j)\right).
\end{eqnarray}
Here
\begin{equation}
\frac{\mathrm{d}D_s}{\mathrm{d}R}(R_i) =
\frac{(1+\epsilon_s)k\alpha_i\exp(-\alpha_iR_i)}{\left(1+k\exp(-\alpha_iR_i)\right)^2}.
\end{equation}

This can be written in compact form
\begin{equation}
\label{dscompact}\frac{\mathrm{d}D_s}{\mathrm{d}R}(R_i) =
\frac{\alpha_i}{1+\epsilon_s}(1+D_s(R_i))(\epsilon_s-D_s(R_i)).
\end{equation}

\subsection*{Self energy force.}

To obtain the force we need to find
\begin{equation}
\frac{\mathrm{d}r_{ij}}{\mathrm{d}\mathbf{r}_i} =
\frac{\mathrm{d}||\mathbf{r}_j-\mathbf{r}_i||}{\mathrm{d}\mathbf{r}_i}~=~-\frac{||\mathbf{r}_j-\mathbf{r}_i||}{r_{ij}}~=~-\hat{\mathbf{r}}_{ij},
\end{equation}
and similarly for $\mathbf{r}_j$, then apply the chain rule to get
\begin{equation}
\label{nablaes}\nabla_{ij}E_s ~=~ \frac{\delta E_s}{\delta r_{ij}} ~[-\hat{\mathbf{r}}_{ij} ~~ \hat{\mathbf{r}}_{ij}~].
\end{equation}

For the force
\begin{equation}
\mathbf{F}_{ij} = -\nabla_{ij}E_s ~=~ \frac{\delta E_s}{\delta r_{ij}} ~[~\hat{\mathbf{r}}_{ij} ~ -\hat{\mathbf{r}}_{ij}~].
\end{equation}


\subsection*{Self energy second derivative.}
The second derivative of the self energy is
\begin{eqnarray}
\nonumber\frac{\delta^2 E_s}{\delta r_{ij}^2} &=&
\frac{q_i^2}{R_i}\left[
\left(\frac{\mathrm{d}R_i}{\mathrm{d}r_{ij}}\right)^2\left(
\left(\frac{\mathrm{d}D_s}{\mathrm{d}R}\right)^2\frac{1}{D_s^3}
-\frac{1}{2}\frac{\mathrm{d}^2D_s}{\mathrm{d}R^2}\frac{1}{D_s^2}
+\frac{\mathrm{d}D_s}{\mathrm{d}R}\frac{1}{D_s^2R_i}
+\frac{1}{D_sR_i^2}-\frac{1}{R_i^2}
\right)\right.\\
\nonumber&&\quad\left.\left.-\frac{1}{2}\frac{\mathrm{d}^2R_i}{\mathrm{d}r_{ij}^2}
\left(
\frac{\mathrm{d}D_s}{\mathrm{d}R}\frac{1}{D_s^2}
+\frac{1}{D_sR_i}
-\frac{1}{R_i}
\right)
\right]\right|_{R_i}\\
\nonumber&&\quad\quad+\frac{q_j^2}{R_j}\left[
\left(\frac{\mathrm{d}R_j}{\mathrm{d}r_{ij}}\right)^2\left(
\left(\frac{\mathrm{d}D_s}{\mathrm{d}R}\right)^2\frac{1}{D_s^3}
-\frac{1}{2}\frac{\mathrm{d}^2D_s}{\mathrm{d}R^2}\frac{1}{D_s^2}
+\frac{\mathrm{d}D_s}{\mathrm{d}R}\frac{1}{D_s^2R_j}
+\frac{1}{D_sR_j^2}-\frac{1}{R_j^2}
\right)\right.\\
&&\quad\quad\quad\left.\left.-\frac{1}{2}\frac{\mathrm{d}^2R_j}{\mathrm{d}r_{ij}^2}
\left(
\frac{\mathrm{d}D_s}{\mathrm{d}R}\frac{1}{D_s^2}
+\frac{1}{D_sR_j}
-\frac{1}{R_j}
\right)
\right]\right|_{R_j},
\end{eqnarray}
where `evaluated at' denotes the argument used in function $D_s$ and its derivatives.

Here the second derivative of the screening function is
\begin{equation}
\frac{\mathrm{d}^2D_s}{\mathrm{d}R^2}(R_i) = \frac{2(1+\epsilon_s)k^2\alpha_i^2\exp(-2\alpha_iR_i)}{\left(1+k\exp(-\alpha_iR_i)\right)^3} -
\frac{(1+\epsilon_s)k\alpha_i^2\exp(-\alpha_iR_i)}{\left(1+k\exp(-\alpha_iR_i)\right)^2}.
\end{equation}
From Eqn.(\ref{dscompact}) we can write this more compactly
\begin{equation}
\frac{\mathrm{d}^2D_s}{\mathrm{d}R^2}(R_i) = \frac{\alpha_i}{1+\epsilon_s}
\left(\epsilon_s-1-2D_s(R_i)\right)\frac{\mathrm{d}D_s}{\mathrm{d}R}(R_i).
\end{equation}

The second derivative of the Bourn radius
\begin{equation}
\frac{\mathrm{d}^2R_j}{\mathrm{d}r_{ij}^2} =
\eta_i\left(
\frac{\mathrm{d}^2f}{\mathrm{d}r_{ij}^2}(r_{ij})
-2\frac{\mathrm{d}f}{\mathrm{d}r_{ij}}(r_{ij})C_i
+f(r_{ij})C_i^2
\right)\exp(-C_ir_{ij}),
\end{equation}
where, for $r_{ij} < 5$,
\begin{equation}
\frac{\mathrm{d}^2f}{\mathrm{d}r_{ij}^2}(r_{ij}) =
r_{ij}^2\left(1-\frac{r_{ij}^4}{625}\right)^2\left(
\frac{192}{390625}r_{ij}^4
-\frac{48}{625}\left(1-\frac{r_{ij}^4}{625}\right)
\right).
\end{equation}

For polar Hydrogen we have the additional term
\begin{equation}
\frac{\mathrm{d}^2R_i^{H+}}{\mathrm{d}r_{ij}^2} =
\frac{\mathrm{d}^2R_i}{\mathrm{d}r_{ij}^2} +
g_ig_j\left(\frac{\mathrm{d}^2f}{\mathrm{d}r_{ij}^2}(r_{ij})-2\frac{\mathrm{d}f}{\mathrm{d}r_{ij}}(r_{ij})E_i+f(r_{ij})E_i^2\right)\exp(-E_ir_{ij}).
\end{equation}

\subsection*{Self energy Hessian.}

We can now differentiate Eq. (\ref{nablaes}) w.r.t. $\mathbf{r}_{ij}$ to get the Hessian for atoms $i$ and $j$.
\begin{equation}
\mathbf{H}_{ij}=
\frac{\delta E_s}{\delta r_{ij}}\frac{1}{r_{ij}}
\left[
\begin{array}{r r}
\mathbf{I} & -\mathbf{I}\\
\\
-\mathbf{I} & \mathbf{I}
\end{array}\right]
+
\left(
\frac{\delta^2 E_s}{\delta r_{ij}^2}-\frac{\delta E_s}{\delta r_{ij}}\frac{1}{r_{ij}}
\right)
\left[
\begin{array}{r r}
\hat\mathbf{r}_{ij}\hat\mathbf{r}_{ij}^{\mathrm{T}} & -\hat\mathbf{r}_{ij}\hat\mathbf{r}_{ij}^{\mathrm{T}}\\
\\
-\hat\mathbf{r}_{ij}\hat\mathbf{r}_{ij}^{\mathrm{T}} & \hat\mathbf{r}_{ij}\hat\mathbf{r}_{ij}^{\mathrm{T}}
\end{array}\right].
\end{equation}

\section*{Original switch function.}

In the papers by Hassan the original switch function was given by
\begin{equation}
f(r_{ij}) = \left\{
\begin{array}{c c}
\left(1-\frac{r_{ij}^2}{25}\right)^2 & 0\le r_{ij}<5,\\
0 & r_{ij}\ge 5.
\end{array}
\right.
\end{equation}
For $r_{ij} < 5$ this switch has first derivative,
\begin{equation}
\frac{\mathrm{d}f}{\mathrm{d}r_{ij}}(r_{ij}) =
-\frac{4}{25}r_{ij}\left(1-\frac{r_{ij}^2}{25}\right),
\end{equation}
and second derivative,
\begin{equation}
\frac{\mathrm{d}^2f}{\mathrm{d}r_{ij}^2}(r_{ij}) =
-\frac{12}{625}r_{ij}^2 -\frac{4}{25}.
\end{equation}












\end{document}
%
% ****** End of file ******
